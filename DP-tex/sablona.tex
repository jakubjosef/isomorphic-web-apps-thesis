
\chapter[Šablona pro závěrečné práce ]{Šablona závěrečné práce pro \LyX{} a~\LaTeX{} }

Šablona%
\footnote{Tato šablona (její zdrojové soubory) je ke stažení na \texttt{http://people.fsv.cvut.cz
/www/vydra/files/diplomka-bakalarka-lyx-sablona.zip}. Další skvělé
šablony pro závěrečné práce viz~\cite{Thesis-templates,Diplomka-v-LaTeXu,Jirkovy-stranky,diplPraceSLU,Vavreckova}. %
} \texttt{bakalarka.lyx} je vytvořena v editoru \LyX{}%
\footnote{Což je\emph{ frontend} \LaTeX{}u.%
}, je určena pro editor \LyX{}, ale byla též vyexportována pro použití
v jiných editorech pro \LaTeX{} jako \texttt{bakalarka.tex}. Od ostatních
\emph{\LaTeX{}ových} šablon se odlišuje především tím, že se zaměřuje
na kvalitu použitého písma. 

Šablona je založena na výborné \LaTeX{}ové  třídě \emph{KOMA-Script
report (scrrpt)}, vycházející z evropských (zvláště německých) typografických
zvyklostí, které se od českých odlišují jen minimálně\emph{. }Byla
by škoda neuměle zasahovat do typograficky dokonalého díla, a proto
v základním nastavení třídy nebyly provedeny téměř žádné změny. Šablona
zavádí kvalitní\emph{ česká} písma\emph{}%
\footnote{Tím jsou míněna písma s kvalitní českou diakritikou, nikoliv českého
původu. Většina použitých písem je polské provenience.\emph{ }%
} a s nimi sladěné matematické symboly, pestřejší záhlaví stránek a
pestřejší hlavičky kapitol. Podrobněji u jednotlivých balíčků (část~\ref{sec:Zm=00011Bna-vzhledu}).

\begin{description}
\item [{Důležité~upozornění:}] Šablona je určena pro \emph{oboustranné}
vytištění!%
\marginpar{\textbf{\Huge }%
\ovalbox{\textbf{\Huge !}}%
}
\end{description}

\section{Struktura šablony}

Šablona je vytvořena tak, abyste se v ní při psaní rozsáhlé práce
mohli pokud možno dobře orientovat. Každá jednotlivá kapitola je proto
uložena ve zvláštním souboru pojmenovaném podle názvu kapitoly. Hlavní
soubor, který vše zastřešuje a spojuje do jednoho celku je \texttt{bakalarka.lyx}.
Jednotlivé kapitoly je možné editovat samostatně a z každé z nich
je možné vytvořit pdf náhled (nelekněte se, že místo odkazů směřujících
na jiné kapitoly budou otazníky). Celou práci zkompilujete vytvořením
pdf ze souboru \texttt{bakalarka.lyx. }Další kapitoly lze přidat z
hlavního menu \LyX{}u: \texttt{Vložit $\rightarrow$ Soubor $\rightarrow$
Dokument potomka}.


\section{Změna vzhledu \label{sec:Zm=00011Bna-vzhledu}}

Tuto část doporučuji číst až po dopsání práce anebo ji raději nečíst
vůbec. Dočtete se v ní pouze jak změnit celkový vzhled práce, tj.
jak změnit písmo, záhlaví stránek a hlavičky kapitol. Pokud vám vzhled
práce vyhovuje, tak jak je nastaven, nemusíte měnit ani nastavovat
vůbec nic! Změna vzhledu se provádí v preambuli \LaTeX{}u pomocí tzv.
balíčků (viz~\ref{sub:Zavedeni}), a je pro začátečníka poněkud nezvyklá
.


\subsection{Písma}

Balíčky (z anglického packages) s písmy jsou pojmenovány podle typu
základního písma, i když zavádí i písmo vedlejší. Vedlejší písmo je
obvykle písmo bezpatkové (bezšerifové), které se většinou používá
na nadpisy všech úrovní. Další vedlejší písmo je strojopis, který
se používá při výpisech kódu programů, ke zvýraznění internetových
adres, názvů počítačových souborů atp. Každý balíček též zavádí matematické
symboly sladěné se základním písmem.

\begin{description}
\item [{bc-latinmodern}] -- standardní \LaTeX{}ová písma Latin Modern.
Matematické symboly \`ala Latin Modern.
\item [{bc-times}] -- písma z balíčku \TeX-Gyre. Základní Termes, vedlejší
Heros, strojopis Cursor. Matematika \`ala times.
\item [{bc-palatino}] -- písma z balíčku \TeX-Gyre. Základní: Pagella,
vedlejší Heros, strojopis Cursor. Matematika: \`ala palatino
\item [{bc-iwona}] -- základní i vedlejší písmo bezpatkové písmo Iwona,
strojopis  Cursor. Matematika: \`ala Iwona.
\item [{bc-helvetika}] -- základní i vedlejší písmo bezpatkové písmo \TeX-Gyre
Heros, strojopis Cursor. Matematika: \`ala helvetika.
\end{description}

\subsection{Záhlaví stránek}

\begin{description}
\item [{bc-headings}] -- záhlaví základním písmem, bez linky.
\item [{bc-fancyheaders}] -- záhlaví stránek z \LaTeX{}ového balíčku \emph{fancyhdr}
-- pestré, kapitálky základního písma, s linkou.
\end{description}

\subsection{Hlavičky kapitol}

\begin{description}
\item [{bc-neueskapitel}] -- hlavičky \emph{Neues Kapitel}~\cite{NeuesKapitel}.
\item [{bc-fancychap}] \textbf{-- }hlavičky kapitol z \LaTeX{}ového balíčku
\emph{fncychap}%
\footnote{Editací souboru \texttt{bc-fancychap.sty }lze vybrat ze\emph{ sedmi}
různých  hlaviček kapitol. Některé hlavičky si mění písmo -- je třeba
zkontrolovat!%
}. 
\end{description}

\subsection{Zavedení balíčků pro změnu vzhledu\label{sub:Zavedeni}}

Balíčky se zavádějí v preambuli \LaTeX u (v dokumentu \texttt{bakalarka.lyx}!)
v hlavním menu \LyX{}u: \texttt{Dokument $\rightarrow$ Nastavení\ldots{}
$\rightarrow$ Preabule LaTeXu}, v textovém poli napravo odstraníte
znak \% před příkazem volajícím příslušný balíček. Tedy před \texttt{\textbackslash{}usepackage\{packages/bc-times\}},
pokud chceme použít balíček bc-times (zavádí písma typu times + helvetika
+ curier) . Před ostatními příkazy musí znak \% zůstat, nebo ho tam
doplňte:

\begin{lyxcode}
{\footnotesize \%<-{}-{}-{}-{}-{}-{}-{}-{}-{}-{}-{}-volání~balíčků-{}-{}-{}-{}-{}-{}-{}-{}-{}-{}-{}-{}-{}-{}-{}-{}-{}-{}-{}-{}->}{\footnotesize \par}

{\footnotesize \%~(znak~\%~je~označení~komentáře:~co~je~za~ním,~není~aktivní}{\footnotesize \par}

{\footnotesize \%<-{}-{}-{}-{}-{}-{}-{}-{}-{}-{}-{}-{}-písmo-{}-{}-{}-{}-{}-{}-{}-{}-{}-{}-{}-{}-{}-{}-{}-{}-{}-{}-{}-{}-{}-{}-{}-{}-{}-{}->~}{\footnotesize \par}

{\footnotesize \%\textbackslash{}usepackage\{packages/bc-latinmodern\}\%~zavádí~písmo~~a~mat}{\footnotesize \par}

{\footnotesize{}~\textbackslash{}usepackage\{packages/bc-times\}\%~zavádí~písma~times,~helve}{\footnotesize \par}

{\footnotesize \%\textbackslash{}usepackage\{packages/bc-palatino\}\%~zavádí~písma~palatino,~}{\footnotesize \par}

{\footnotesize \%\textbackslash{}usepackage\{packages/bc-iwona\}\%zavádí~písmo~~iwona}{\footnotesize \par}
\end{lyxcode}
Balíčky jsou definovány v~souborech s příponou \texttt{sty}, což
jsou obyčejné textové soubory a lze je editovat%
\footnote{Myslím, že pochopit, který příkaz co znamená, není tak těžké, takže
lze snadno vytvořit ,,hybridní{}`` balíčky.%
} (jsou uloženy ve složce \texttt{packages}).

\begin{lyxcode}

\end{lyxcode}

\subsection{Jaký vzhled vybrat?}

To nechám na vás. Nechcete-li riskovat, volte konzervativní vzhled:
bc-times + bc-headings nebo bc-fancyheaders, hlavičky kapitol žádné
(základní).


\subsection*{Příklady}

\begin{itemize}
\item Zkompilujete-li tlačítkem \includegraphics{obrazky/pdf} \emph{celý
tento dokument} (tj. originální soubor \texttt{bakalarka.lyx}), bude
text vysázen písmem \TeX{}-Gyre Pagella, hlavičky písmem \TeX{}-Gyre
Heros, záhlaví bude \emph{bc-fancyhdr }a\emph{ }hlavičky kapitol \emph{Fancychap}
(Bjarne).
\item Zkompilujete-li samostatně \emph{tuto }kapitolu (originální soubor
\texttt{sablona.lyx}) bude text vysázen písmem \TeX{}-Gyre Thermes,
hlavičky písmem \TeX{}-Gyre Heros, záhlaví bude \emph{fancyhdr. }Hlavička
kapitoly je \emph{Neues Kapitel}.
\item Zkompilujete-li samostatně\emph{ }kapitolu \emph{Úvod} (originální
soubor \texttt{uvod.lyx}) budou text i hlavičky vysázeny písmem Iwona.\emph{
}Hlavičky kapitol jsou \emph{základní}.
\item Zkompilujete-li samostatně\emph{ }kapitolu \emph{Úvod do \LyX{}u}
(originální soubor \texttt{uvod\_L}\-\texttt{yX.lyx}) budou text
i hlavičky vysázeny písmem Latin Modern.\emph{ }Hlavičky kapitol jsou
\emph{základní}.
\item Zkompilujete-li samostatně\emph{ }kapitolu \emph{Závěr} (originální
soubor \texttt{zaver.lyx}) budou text i hlavičky vysázeny písmem \TeX{}-Gyre
Heros.\emph{ }Hlavička kapitoly je \emph{Fancychap} (Bjarne).
\end{itemize}

