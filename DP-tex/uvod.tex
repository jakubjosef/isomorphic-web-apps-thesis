\pagenumbering{arabic}%start arabic pagination from 1 

\chapter{Úvod}
S dlouhodobým rozmachem internetu se zvyšuje i komplexita webových aplikací. Statické webové prezentace, pro které byl protokol HTTP navržen, už dnes v podstatě neexistují. Nahradily je složité dynamické aplikace, využívající jednotky, desítky nebo dokonce stovky počítačových serverů, obsluhující velké množství uživatelů. S rostoucími požadavky a tlakem na klesání nákladů na vývoj moderních webových aplikací vznikají ucelené knihovny a typová řešení usnadňující jejich tvorbu. Mezi jednu z oblastí knihoven pro vývoj webových stránek patří také webové frameworky. Cílem každého frameworku je usnadnit programování nějaké aplikace. Webové frameworky poskytují standardní rozhraní především pro:
\begin{itemize}
\item zpracování HTTP požadavků,
\item generování HTML,
\item obsluhu komunikace s databází,
\item přístup k souborovému systému,
\item externí komunikaci.
\end{itemize}

Takových frameworků dnes existuje celá řada pro všechny programovací jazyky, které se běžně používají k tvorbě webových aplikací. Většina existujících řešení vzniklo nad jazyky jako PHP, ASP.NET, Python nebo Ruby. Základním paradigmatem těchto řešení, je princip přijmutí HTTP requestu, jeho zpracování, odeslání odpovědi webovému prohlížeči a ukončení spojení. Tato řešení jsou nazývána jako \textit{serverové webové frameworky}. Bezstavovost celé komunikace mezi uživatelem a serverem vychází s návrhu protokolu HTTP, na tomto chování tedy není nic špatného, avšak možnost využívaní stavů je u dynamické webové aplikace téměř vždy nezbytné. Bez konceptů do značné míry simulující stavovost by nebylo snadno možné implementovat například přihlašování uživatelů. Stavovost HTTP requestů je simulována především pomocí \textit{sessions} a \textit{cookies}. Jedná se zpravidla o jedinečný identifikátor spojení, podle kterého webový server dokáže takové HTTP spojení identifikovat a přiřadit jej ke konkrétnímu uživateli. Tyto mechanismy dokázaly efektivně provozovat dynamickou webovou aplikaci nad protokolem HTTP víc než dvacet let. Nutnost načtení celé HTML stránky při každém požadavku uživatele byla standardem, na který byli uživatelé zvyklí. V posledních letech můžeme pozorovat nárůst oblíbenosti velmi interaktivních webových aplikací, kde již tento princip neplatí. Moderní aplikace jsou schopné obnovovat jen ty části stránky, které jsou zrovna potřeba a pracovat s daty na pozadí. Jediným jazykem umožňujícím provádět operace s webovou stránkou uvnitř webového prohlížeče, a tím realizovat změny bez nutnosti načtení nové stránky, je Javascript. Pojem vývoj webových aplikací, který do té doby obsahoval znalost 3 hlavních programovacích jazyků, HTML, CSS a jakéhokoliv dynamického serverového jazyka, začal registrovat další jazyk, bez kterého se již webový výboj neobejde. Dnes je požadavek na alespoň elementární znalost Javascriptu a jeho nejznámějších knihoven, součástí téměř každé nabídky pracovní pozice webového programátora. Typický webový vývojář je tedy při práci nucen využívat 4 nebo i více programovacích jazyků současně. Na serveru existuje mnoho programovacích jazyků, ve kterých lze aplikaci psát, zatímco webový prohlížeč zná jediný: Javascript. Bezpochyby každý člověk, který používá internet se jistě vědomě nebo nevědomě setkal s tímto jazykem. Tento jazyk je jednou se základních součástí většiny webových stránek současnosti. Doby, kdy bylo na webu nutné počítat s vypnutým Javascriptem u některých uživatelů, jsou nenávratně pryč. Většina populárních webových stránek, jako je Google nebo Facebook, už bez Javascriptu nefunguje nebo funguje velmi omezeně, Javascript se stal nezbytnou součástí moderního webu. Méně známější je fakt, že lze tento jazyk používat také mimo internetový prohlížeč. Lze ho uplatit také na webovém serveru, v rozšířeních pro prohlížeče nebo v mobilních či desktopových aplikacích. Jeho využití ve webových aplikacích neustále stoupá a díky platformě node.js je možné používat Javascript i jako serverový jazyk. Je tedy možné využívat stejný jazyk pro prohlížeč i pro server nebo mezi nimy dokonce sdílet kód. Tomuto přístupu se začalo říkat isomorfní (vzájemně jednoznačné) webové aplikace v jazyce Javascript, o kterých pojednává i tato práce.

\section{Cíl, metodika a předpoklady práce}
Cílem této práce je popsat jazyk Javascript a technologií na něm založených nebo s ním souvisejících, a to především těch, které splňují princip isomorfního programování. Isomorfismus v tomto kontextu znamená použití stejného jazyka pro server i klient, u webových aplikací je vhodným kandidátem jazyk Javascript. Práce stručně shrnuje jeho historii a především bouřlivý vývoj několika posledních let. Představí nový standard ECMAScript 6, který přináší významnou evoluci tohoto jazyka. Hlavní novinky toho standardu budou představeny spolu s příklady jejich využití. Dále práce popíše jednostránkové webové aplikace (SPA), spolu se souvisejicími historickými milníky, které vedly ke jejich vzniku a typickými architekturami, díky kterým je možné tento princip webových stránek využívat. Druhým hlavním cílem je představit stěžejní principy isomorfního přístupu k tvorbě webových aplikací spolu s jejich hlavními výhodami či nevýhodami vzhledem k zažitým zvyklostem u jiných programovacích jazyků. Práce srovnává moderní isomorfní přístup s časem ověřenými metodami vývoje webových aplikací z několika hledisek. Především je to z hlediska použitelnosti v prostředí webu, hlavních výhod či nevýhod zmíněných přístupů. Také je porovnána náročnost implementace a složitost přípravy vývojového prostředí.

V práci jsou předpokládány alespoň základní znalosti z oblasti vývoje webových aplikací a alespoň elementární znalost jazyka Javascript. Vedle českých zažitých výrazů jsou použity také anglické termíny, a to tehdy, nenašel-li se zatím pro termín ustálený český ekvivalent. Grafy, diagramy a ukázky zdrojových kódů používají výhradně anglická pojmenování.

\section{Struktura práce}
Práce se fakticky dělí na dvě základní části. V teoretické části je představen programovací jazyk Javascript, spolu s jeho krátkou historií a vývojem v posledních letech. Čtenář se dozví o mohutném vývoji tohoto jazyka, nejen v prostředí webových prohlížečů, ale také serverů a o dalších možnostech jeho využití. Dále je představen pojem jednostránková webová aplikace, který je v poslední době velice aktuální a úzce souvisí s isomorfním přístupem k webovému vývoji. Následující kapitola se věnuje popisu samotného isomorfního přístupu, spolu s jeho hlavními výhodami a souvisejícími důsledky pro uživatele.
Budou také shrnuty nejpoužívanější knihovny, vhodné pro tuto oblast vývoje. 
Praktickou část tvoří ukázková isomorfní webová aplikace, na které jsou demonstrovány základní principy isomorfního přístupu spolu s přehledem použitých návrhových vzorů a vhodných knihoven. Na závěr práce bude porovnán isomorfní přístup se zažitými metodami vývoje webových aplikací a nastíněn další možný výzkum v oblasti.

\section{Literární rešerše v oblasti}
Téma frameworků pro vývoj webových aplikací je velmi široké a hojně řešené. Většina existujících prací se zaměřuje hlavně na typické serverové frameworky, popisuje jejich návrhové vzory, porovnává výhody různých přístupů v různých programovacích jazycích, nebo měří rychlost zpracování a generování HTML kódu. Práce zabývající se vývojem webových aplikací v jazyce Javascript, začínají vycházejí ve velkém počtu až v posledních několika letech. Převážně se týkají jednostránkových webových aplikací v Javascriptu. Těm se věnuje mnoho prací, v České republice třeba Šimon Mareš \cite{simon_spa} nebo Marek Horyna \cite{spa_horyna}. Pojem isomorfní webová aplikace se poprvé objevil v roce 2011 v článku \textit{Scaling Isomorphic Javascript Code} \cite{isomorphic_founder} od Charlieho Robbinse. Ten se zamýšlí nad zvýšením výkonu javascriptových aplikací a představuje myšlenku většího zapojení webového serveru než u tradičních jednostránkových aplikací. Princip isomorfismu se začal používat v praxi kolem roku 2015, zatím se mu ale věnovalo jen velmi málo akademických prací. Jednou z nich, která ho dobře popisuje, je například Isomorphic web applications - Depends on how you react od Erica Matthiasona \cite{mathiasson-isomorphic}. Ten se věnuje popisu hlavních výhod isomorfního přístup a historických milníků, které vedly ze jeho vzniku. Závěrem se jeho práce zabývá porovnáním javascriptových frameworků React a Ember.js. Ve stejném roce vyšla také kniha \textit{Building Isomorphic JavaScript Apps From Concept to Implementation to Real-World Solutions} od autorů Jason Strimpel a Najim Maxime \cite{isomorhic_book}, která se také zabývá přestavením isomorfního návrhu webových aplikací, spolu s popisem doporučených knihoven. Publikace obsahuje také mnoho ukázek zdrojového kódu v jazyce Javascript.


